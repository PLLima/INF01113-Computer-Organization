\documentclass{beamer}
\usetheme{Madrid}
\usecolortheme{seahorse}
\setbeamertemplate{caption}[numbered]

\usepackage{lmodern}

% Language setting
\usepackage[main=brazilian, english]{babel}

\usepackage{tcolorbox}
\usepackage{listings}
\usepackage{adjustbox}
\usepackage{tikz}
\usepackage{pgfplots}
\pgfplotsset{
    table/search path={./data},
    compat=newest
}

\usepackage{caption}

\usepackage{ragged2e}
\usepackage{etoolbox}
\usepackage{lipsum}

\apptocmd{\frame}{}{\justifying}{}

\setbeamerfont{title}{series=\bfseries}
\setbeamerfont{subtitle}{series=\normalfont}
\defbeamertemplate*{title page}{customized}[1][]
{
    \vbox{}
    \vfill
    \begingroup
        \centering
        \usebeamertemplate{titlegraphic}
        \includegraphics[width=0.2\linewidth]{images/logos/UFRGS.png}
        \hspace{0.5cm}
        \includegraphics[width=0.2\linewidth]{images/logos/INF.png}
        \par
        
        \vskip 0.15cm

        \usebeamerfont{institute}
        \insertinstitute
        \par

        \vskip 0.5cm

        \begin{tcolorbox}[colframe=darkgray, colback=lightgray]

            \centering
            \usebeamerfont{title}
            \inserttitle
            \par

            \usebeamerfont{subtitle}
            \insertsubtitle
            \par

        \end{tcolorbox}

        \vskip 0.5cm

        \usebeamerfont{author}
        \insertauthor
        \par

        \vskip 0.5cm

        Turma B
        \par

    \endgroup
    \vfill
}

\title[Segundo Trabalho]{Segundo Trabalho}
\subtitle{\textit{Benchmarks} de Organizações de Computadores}
\author[Grupo 13]{Bruno Alexandre Hofstetter Bourscheid (00550177)
              \\ Fernando Longhi de Andrade (00580366)
              \\ Luiz Augusto Ponzoni Schmidt (00580108)
              \\ Miguel Dutra Fontes Guerra (00342573)
              \\ Pedro Lubaszewski Lima (00341810)}
\institute[]{INF01113\\Organização De Computadores B}
\date[\today]{\today}

\renewcommand{\lstlistingname}{Código}

\begin{document}

    \begin{frame}
        \maketitle
    \end{frame}

    \begin{frame}
        \frametitle{Algoritmos Selecionados}
        \begin{columns}
            \begin{column}{0.33\textwidth}

                \begin{center}
                    \textbf{\textit{Bubble Sort}}
                \end{center}

                \begin{itemize}
                    \footnotesize
                    \item Complexidade: $O(n^2)$;
                    \item Algoritmo de ordenação simples;
                    \item Muitas comparações e trocas;
                    \item Baixo uso de memória;
                    \item Pouco paralelismo (dependência entre as instruções).
                \end{itemize}

            \end{column}

            \begin{column}{0.01\textwidth}
                \rule{.1mm}{0.75\textheight}
            \end{column}

            \begin{column}{0.33\textwidth}

                \begin{center}
                    \textbf{\textit{Fast Fourrier Transform} (FFT)}
                \end{center}

                \begin{itemize}
                    \footnotesize
                    \item Complexidade: $O(n\log(n))$;
                    \item Operações matemáticas complexas;
                    \item Várias instruções de ponto flutuante;
                    \item Acesso regular à memória;
                    \item Alto paralelismo.
                \end{itemize}

            \end{column}

            \begin{column}{0.01\textwidth}
                \rule{.1mm}{0.75\textheight}
            \end{column}

            \begin{column}{0.33\textwidth}

                \begin{center}
                    \textbf{Multiplicação de Matrizes}
                \end{center}

                \begin{itemize}
                    \footnotesize
                    \item Complexidade: $O(n^3)$;
                    \item Poucas instruções;
                    \item Bastante acesso à memória contígua (depende da \textit{cache});
                    \item Depende da latência de acesso à memória.
                \end{itemize}

            \end{column}
        \end{columns}
    \end{frame}

    \begin{frame}
        \frametitle{Parâmetros de Organização}

        \begin{columns}
            \begin{column}{0.3\textwidth}

                \begin{center}
                    \begin{tikzpicture}
                        \node[circle, fill=blue!70, draw, minimum size=.75cm] (node1) {1};
                    \end{tikzpicture}

                    \Large
                    \textbf{Tamanho da \textit{Cache} L1}
                \end{center}

                \justifying
                \footnotesize
                Representa a quantidade de dados mais próximos da \textit{CPU}. Isso pode afetar o desempenho
                de programas que dependam de dados com localidade temporal e espacial, afetando o número de
                \textit{cache misses}.

            \end{column}

            \begin{column}{0.3\textwidth}

                \begin{center}
                    \begin{tikzpicture}
                        \node[circle, fill=blue!70, draw, minimum size=.75cm] (node1) {2};
                    \end{tikzpicture}

                    \Large
                    \textbf{Associatividade da \textit{Cache} L1}
                \end{center}

                \justifying
                \footnotesize
                É o número de blocos de cada conjunto de memória \textit{cache}. Também deve afetar a questão
                de programas com dados que apresentam localidade espacial, através de mais ou menos \textit{cache misses}.

            \end{column}

            \begin{column}{0.3\textwidth}

                \begin{center}
                    \begin{tikzpicture}
                        \node[circle, fill=blue!70, draw, minimum size=.75cm] (node1) {3};
                    \end{tikzpicture}

                    \Large
                    \textbf{Tamanho do \textit{Fetch Buffer}}
                \end{center}

                \justifying
                \footnotesize
                Um maior tamanho de \textit{fetch buffer} implica em mais instruções sendo processadas simultâneamente. Ou seja,
                deve afetar algoritmos que apresentem maior paralelismo de instruções.

            \end{column}
        \end{columns}

    \end{frame}

    \begin{frame}
        \frametitle{Configuração Fixa}

        \LARGE
        Como configuração fixa inicial, utilizou-se os seguintes parâmetros:

        \begin{itemize}
            \item Tamanho da \textit{Cache} L1: 16kB;
            \item Associatividade da \textit{Cache} L1: 8-\textit{way};
            \item Tamanho do \textit{Fetch Buffer}: 64\textit{bytes};
        \end{itemize}
        Os demais parâmetros de configuração da máquina foram mantidos.

    \end{frame}

    \begin{frame}
        \frametitle{Resultados mudando o tamanho de \textit{cache} L1}

        \begin{columns}
            \begin{column}{0.5\textwidth}
                \centering
                \begin{tikzpicture}[scale=0.7]
                    \pgfplotsset{
                        every axis legend/.append style={ at={(0.7,-0.4)}, anchor=center,legend columns = 1},
                    }
                    \begin{axis}[
                        title={IPC},
                        xlabel={Tamanho da \textit{Cache} L1 [kB]},
                        ylabel={Instruções por Ciclo},
                        xmin=1, xmax=17,
                        ymin=0.8, ymax=1.3,
                        xtick={2,4,8,16},
                        ymajorgrids=true,
                        xmajorgrids=true,
                        grid style=dashed,
                    ]
                        \addplot table[
                            col sep=comma,
                            row sep=newline,
                            color=blue,
                            mark=circle,
                            x=size,
                            y=bubblesort
                        ]
                        {CacheSize_IPC.csv};
                        \addlegendentry{\textit{Bubble Sort}};

                        \addplot table[
                            col sep=comma,
                            row sep=newline,
                            color=red,
                            mark=circle,
                            x=size,
                            y=fft
                        ]
                        {CacheSize_IPC.csv};
                        \addlegendentry{FFT};

                        \addplot table[
                            col sep=comma,
                            row sep=newline,
                            color=green,
                            mark=circle,
                            x=size,
                            y=matrix-mult
                        ]
                        {CacheSize_IPC.csv};
                        \addlegendentry{Mult. de Matrizes};

                        \addplot table[
                            col sep=comma,
                            row sep=newline,
                            color=black,
                            mark=circle,
                            style=densely dashed,
                            x=size,
                            y expr={(\thisrow{bubblesort}+\thisrow{fft}+\thisrow{matrix-mult})/3}
                        ]
                        {CacheSize_IPC.csv};
                        \addlegendentry{Média};
                    \end{axis}
                \end{tikzpicture}
            \end{column}

            \begin{column}{0.5\textwidth}
                \centering
                \begin{tikzpicture}[scale=0.7]
                    \pgfplotsset{
                        every axis legend/.append style={ at={(0.7,-0.4)}, anchor=center,legend columns = 1},
                    }
                    \begin{axis}[
                        title={Tempo de Execução},
                        xlabel={Tamanho da \textit{Cache} L1 [kB]},
                        ylabel={Tempo [s]},
                        xmin=1, xmax=17,
                        ymin=0.7, ymax=1.4,
                        xtick={2,4,8,16},
                        ymajorgrids=true,
                        xmajorgrids=true,
                        grid style=dashed,
                    ]
                        \addplot table[
                            col sep=comma,
                            row sep=newline,
                            color=blue,
                            mark=circle,
                            x=size,
                            y expr={\thisrow{bubblesort}}
                        ]
                        {CacheSize_time.csv};
                        \addlegendentry{\textit{Bubble Sort}};

                        \addplot table[
                            col sep=comma,
                            row sep=newline,
                            color=red,
                            mark=circle,
                            x=size,
                            y expr={\thisrow{fft}}
                        ]
                        {CacheSize_time.csv};
                        \addlegendentry{FFT};

                        \addplot table[
                            col sep=comma,
                            row sep=newline,
                            color=green,
                            mark=circle,
                            x=size,
                            y expr={\thisrow{matrix-mult}}
                        ]
                        {CacheSize_time.csv};
                        \addlegendentry{Mult. de Matrizes};

                        \addplot table[
                            col sep=comma,
                            row sep=newline,
                            color=black,
                            mark=circle,
                            style=densely dashed,
                            x=size,
                            y expr={(\thisrow{bubblesort}+\thisrow{fft}+\thisrow{matrix-mult})/3}
                        ]
                        {CacheSize_time.csv};
                        \addlegendentry{Média};
                    \end{axis}
                \end{tikzpicture}
            \end{column}
        \end{columns}
    \end{frame}

    \begin{frame}
        \frametitle{Análise dos Resultados}

        \begin{columns}
            \begin{column}{0.5\textwidth}
                \begin{center}
                    \large
                    \textbf{Análise do IPC:}
                \end{center}
                \begin{itemize}
                    \item ;
                    \item ;
                \end{itemize}
            \end{column}

            \begin{column}{0.5\textwidth}
                \begin{center}
                    \large
                    \textbf{Análise do Tempo de Execução:}
                \end{center}
                \begin{itemize}
                    \item ;
                    \item ;
                \end{itemize}
            \end{column}
        \end{columns}
    \end{frame}

    \begin{frame}
        \frametitle{Resultados mudando a associatividade da cache L1}

        \begin{columns}
            \begin{column}{0.5\textwidth}
                \centering
                \begin{tikzpicture}[scale=0.7]
                    \pgfplotsset{
                        every axis legend/.append style={ at={(0.7,-0.4)}, anchor=center,legend columns = 1},
                    }
                    \begin{axis}[
                        title={IPC},
                        xlabel={Associatividade da \textit{Cache} L1 [\#-\textit{way}]},
                        ylabel={Instruções por Ciclo},
                        xmin=0, xmax=17,
                        ymin=0.8, ymax=1.3,
                        xtick={1,2,4,8,16},
                        ymajorgrids=true,
                        xmajorgrids=true,
                        grid style=dashed,
                    ]
                        \addplot table[
                            col sep=comma,
                            row sep=newline,
                            color=blue,
                            mark=circle,
                            x=associativity,
                            y=bubblesort
                        ]
                        {CacheAssociativity_IPC.csv};
                        \addlegendentry{\textit{Bubble Sort}};

                        \addplot table[
                            col sep=comma,
                            row sep=newline,
                            color=red,
                            mark=circle,
                            x=associativity,
                            y=fft
                        ]
                        {CacheAssociativity_IPC.csv};
                        \addlegendentry{FFT};

                        \addplot table[
                            col sep=comma,
                            row sep=newline,
                            color=green,
                            mark=circle,
                            x=associativity,
                            y=matrix-mult
                        ]
                        {CacheAssociativity_IPC.csv};
                        \addlegendentry{Mult. de Matrizes};

                        \addplot table[
                            col sep=comma,
                            row sep=newline,
                            color=black,
                            mark=circle,
                            style=densely dashed,
                            x=associativity,
                            y expr={(\thisrow{bubblesort}+\thisrow{fft}+\thisrow{matrix-mult})/3}
                        ]
                        {CacheAssociativity_IPC.csv};
                        \addlegendentry{Média};
                    \end{axis}
                \end{tikzpicture}
            \end{column}

            \begin{column}{0.5\textwidth}
                \centering
                \begin{tikzpicture}[scale=0.7]
                    \pgfplotsset{
                        every axis legend/.append style={ at={(0.7,-0.4)}, anchor=center,legend columns = 1},
                    }
                    \begin{axis}[
                        title={Tempo de Execução},
                        xlabel={Associatividade da \textit{Cache} L1 [\#-\textit{way}]},
                        ylabel={Tempo [s]},
                        xmin=0, xmax=17,
                        ymin=0.7, ymax=1.4,
                        xtick={1,2,4,8,16},
                        ymajorgrids=true,
                        xmajorgrids=true,
                        grid style=dashed,
                    ]
                        \addplot table[
                            col sep=comma,
                            row sep=newline,
                            color=blue,
                            mark=circle,
                            x=associativity,
                            y expr={\thisrow{bubblesort}}
                        ]
                        {CacheAssociativity_time.csv};
                        \addlegendentry{\textit{Bubble Sort}};

                        \addplot table[
                            col sep=comma,
                            row sep=newline,
                            color=red,
                            mark=circle,
                            x=associativity,
                            y expr={\thisrow{fft}}
                        ]
                        {CacheAssociativity_time.csv};
                        \addlegendentry{FFT};

                        \addplot table[
                            col sep=comma,
                            row sep=newline,
                            color=green,
                            mark=circle,
                            x=associativity,
                            y expr={\thisrow{matrix-mult}}
                        ]
                        {CacheAssociativity_time.csv};
                        \addlegendentry{Mult. de Matrizes};

                        \addplot table[
                            col sep=comma,
                            row sep=newline,
                            color=black,
                            mark=circle,
                            style=densely dashed,
                            x=associativity,
                            y expr={(\thisrow{bubblesort}+\thisrow{fft}+\thisrow{matrix-mult})/3}
                        ]
                        {CacheAssociativity_time.csv};
                        \addlegendentry{Média};
                    \end{axis}
                \end{tikzpicture}
            \end{column}
        \end{columns}
    \end{frame}

    \begin{frame}
        \frametitle{Análise dos Resultados}

        \begin{columns}
            \begin{column}{0.5\textwidth}
                \begin{center}
                    \large
                    \textbf{Análise do IPC:}
                \end{center}
                \begin{itemize}
                    \item ;
                    \item ;
                \end{itemize}
            \end{column}

            \begin{column}{0.5\textwidth}
                \begin{center}
                    \large
                    \textbf{Análise do Tempo de Execução:}
                \end{center}
                \begin{itemize}
                    \item ;
                    \item ;
                \end{itemize}
            \end{column}
        \end{columns}
    \end{frame}

    \begin{frame}
        \frametitle{Resultados mudando o tamanho do \textit{fetch buffer}}


        \begin{columns}
            \begin{column}{0.5\textwidth}
                \centering
                \begin{tikzpicture}[scale=0.7]
                    \pgfplotsset{
                        every axis legend/.append style={ at={(0.7,-0.4)}, anchor=center,legend columns = 1},
                    }
                    \begin{axis}[
                        title={IPC},
                        xlabel={Tamanho do \textit{Fetch Buffer} [\textit{bytes}]},
                        ylabel={Instruções por Ciclo},
                        xmin=1, xmax=72,
                        ymin=0.8, ymax=1.3,
                        xtick={8,16,32,64},
                        ymajorgrids=true,
                        xmajorgrids=true,
                        grid style=dashed,
                    ]
                        \addplot table[
                            col sep=comma,
                            row sep=newline,
                            color=blue,
                            mark=circle,
                            x=buffer-size,
                            y=bubblesort
                        ]
                        {FetchBuffer_IPC.csv};
                        \addlegendentry{\textit{Bubble Sort}};

                        \addplot table[
                            col sep=comma,
                            row sep=newline,
                            color=red,
                            mark=circle,
                            x=buffer-size,
                            y=fft
                        ]
                        {FetchBuffer_IPC.csv};
                        \addlegendentry{FFT};

                        \addplot table[
                            col sep=comma,
                            row sep=newline,
                            color=green,
                            mark=circle,
                            x=buffer-size,
                            y=matrix-mult
                        ]
                        {FetchBuffer_IPC.csv};
                        \addlegendentry{Mult. de Matrizes};

                        \addplot table[
                            col sep=comma,
                            row sep=newline,
                            color=black,
                            mark=circle,
                            style=densely dashed,
                            x=buffer-size,
                            y expr={(\thisrow{bubblesort}+\thisrow{fft}+\thisrow{matrix-mult})/3}
                        ]
                        {FetchBuffer_IPC.csv};
                        \addlegendentry{Média};
                    \end{axis}
                \end{tikzpicture}
            \end{column}

            \begin{column}{0.5\textwidth}
                \centering
                \begin{tikzpicture}[scale=0.7]
                    \pgfplotsset{
                        every axis legend/.append style={ at={(0.7,-0.4)}, anchor=center,legend columns = 1},
                    }
                    \begin{axis}[
                        title={Tempo de Execução},
                        xlabel={Tamanho do \textit{Fetch Buffer} [\textit{bytes}]},
                        ylabel={Tempo [s]},
                        xmin=1, xmax=72,
                        ymin=0.7, ymax=1.4,
                        xtick={8,16,32,64},
                        ymajorgrids=true,
                        xmajorgrids=true,
                        grid style=dashed,
                    ]
                        \addplot table[
                            col sep=comma,
                            row sep=newline,
                            color=blue,
                            mark=circle,
                            x=buffer-size,
                            y expr={\thisrow{bubblesort}}
                        ]
                        {FetchBuffer_time.csv};
                        \addlegendentry{\textit{Bubble Sort}};

                        \addplot table[
                            col sep=comma,
                            row sep=newline,
                            color=red,
                            mark=circle,
                            x=buffer-size,
                            y expr={\thisrow{fft}}
                        ]
                        {FetchBuffer_time.csv};
                        \addlegendentry{FFT};

                        \addplot table[
                            col sep=comma,
                            row sep=newline,
                            color=green,
                            mark=circle,
                            x=buffer-size,
                            y expr={\thisrow{matrix-mult}}
                        ]
                        {FetchBuffer_time.csv};
                        \addlegendentry{Mult. de Matrizes};

                        \addplot table[
                            col sep=comma,
                            row sep=newline,
                            color=black,
                            mark=circle,
                            style=densely dashed,
                            x=buffer-size,
                            y expr={(\thisrow{bubblesort}+\thisrow{fft}+\thisrow{matrix-mult})/3}
                        ]
                        {FetchBuffer_time.csv};
                        \addlegendentry{Média};
                    \end{axis}
                \end{tikzpicture}
            \end{column}
        \end{columns}

    \end{frame}

    \begin{frame}
        \frametitle{Análise dos Resultados}

        \begin{columns}
            \begin{column}{0.5\textwidth}
                \begin{center}
                    \large
                    \textbf{Análise do IPC:}
                \end{center}
                \begin{itemize}
                    \item ;
                    \item ;
                \end{itemize}
            \end{column}

            \begin{column}{0.5\textwidth}
                \begin{center}
                    \large
                    \textbf{Análise do Tempo de Execução:}
                \end{center}
                \begin{itemize}
                    \item ;
                    \item ;
                \end{itemize}
            \end{column}
        \end{columns}
    \end{frame}
\end{document}